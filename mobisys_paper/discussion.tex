\section{Discussion and Future Work}\label{sec:disc}

In this study, we showed that head-movements have the potential to be used as
a reliable biometric characteristic for user authentication.
We will now discuss some of the limitations that we identified from this work
and prospects for future work as below.

\subsection{Reliability}
Our work in this paper shows that head-movements are distinctive and
repeatable in controlled settings. However, in reality, the behavior of
head-movement signatures over chaotic settings will be a key factor to decide
on the effectiveness of this approach. Our work only evaluates the case when a
user is in a stationary setting when attempting to login, such as when sitting
on the chair or standing still. The performance of this approach in realistic
mobile settings such as walking or in a vehicle is yet unknown. It is also
unclear if the head-movement patterns are repeatable in such a mobile environment, or
if the ambiguities of vehicle motion versus head-motion can be separated. 
%%For example, we don't know whether a person's
%%head-movement signature will be the same no matter whether she is sitting,
%%standing still, walking, running, or driving.
%%The most important concern is the reliability of human head-movements. Even
%%though we have shown that head-movements are rather distinctive and
%%repeatable
%%in a controlled setting, it is yet unclear how it will perform in realistic,
%%but chaotic settings. For example, we don't know whether a person's
%%head-movement signature will be the same no matter whether she is sitting,
%%standing still, walking, running, or driving.
Similarly, a person's head-movement signature may also depend on the mood/energy level of
the person; for example, a fresh and energetic user may provide significant
head-movements as compared to a sick or tired user whose signatures may not
even be detectable. Inconsistencies in the accelerometer sensor such as drift and temporal bias can
significantly affect the nature of inferred head-movement signature.
Head-movements, on the other hand, may also evolve over time for a person
which call for periodic calibration of the system and/or the training data.
%To address these two temporal changes, we may need to periodically
%recalibrate
%the sensors and dynamically adjust our training data to reflect new movement
%trends.

While reliability metric was out of the main scope of this paper, we are keen
to address the same in future work.

\subsection{Multi-Modality}
Smart-glass devices typically contain an array of motion sensors such as accelerometer, gyroscope, IMU. It is only a matter of time that motion sensor chips will be integrated into wearable devices.
This opens up opportunities for multi-modal motion sensing. For example,
accelerometer data can be combined with gyroscope measurements to provide
multi-dimensional head-movement features that can improve the quality of the
inferred signatures. Head movements can also be combined with other body movements to generate
valuable, reliable signatures for authentication. 
%Additionally, head-movement is just one type of body movements, and we can
%investigate other types of body movements as well.
For example, through a simple test experiment using the Google Glass
infra-red light sensor\footnote{we had to root the Google Glass to access the
IR sensor unit.} we observed that the blinking and winking patterns of users in
response to the music stimulus were reasonably differentiable among users.
Such patterns may also independently serve as another biometric that can be
used for authentication purpose, or can be combined with head-movements for better results.
Recent studies have shown that heart beat or pulse can also serve as reliable
biometric for authentication purposes~\cite{hernandezbioglass,nymi}
We reserve such potential enhancements to our system for future implementation.
%A recent study has shown that Google glass can detect human heart
%beat~\cite{hernandezbioglass}; heartbeats can thus be
%used as another biometric.
%If extra hardware can be introduced, then more movement patterns,
%such as eye movement, can be leveraged as well.

\subsection{Seamless Protocol}
An authentication system must have an effective protocol for authenticating
users seamlessly to their device. Our system runs a simple authentication
protocol where the user is given a finite set of (calibrated) music
tracks to pick, and based on which the user makes head-movements in response.
Our design assumes that the user voluntarily accepts the enforcement of the
requirement of head-movements in response to the music. A seamless design
would ensure that the system captures even the slightest of the subconscious
head-movements in the event that the user does not make any enforced
head-movements. In such cases the head-movement signatures will have to be
much more elaborate with multiple attributes that correspond to the different
realistic use-cases of the system.
%Our protocol can consist of several steps. In the first step, the user will
%be
%asked to choose a user name from all the legitimate users of the device.
%Then,
%the user will be asked to select the favorite music track of the claimed
%user.
%If the selection is correct, the device will play the music and ask the user
%to move along (including head movement, eye blinking/winking, etc). ***YZ:
%what else??? ***

\subsection{Processing/Battery Power Constraints}
Battery power consumption and computing power are very important parameters
for consideration when optimizing a design to accommodate to wearable devices.
Wearables usually have serious resource limitations, especially in terms of
processing power and battery power.
This paper, addresses these concerns through optimization strategies in the
head-movement signature classification stage of the proposed authentication
algorithm. For wearables it is important that such optimization strategies are
taken to next levels until a roadblock is reached.
%In this paper, we have considered a set of
%optimization techniques to reduce the processing demand as well as power
%consumption, e.g., testing against top $K$ samples instead of  the entire
%training set, using thresholding-based classifier instead of SVM classifier.
For example, one such strategy extension in this work can be that, after a
short duration, before the entire music is played, if it is found that a
user's movement does not match the signature of the claimed user
with sufficient confidence level, then the on-site classification may be
terminated  instead of waiting for the entire duration to yield the rejection.
Another example, may include cyber-foraging strategies to offload heavy
computation tasks, such as classification, to the user's Bluetooth paired
smartphone.

\subsection{Large-scale evaluation}
To be adopted as a primary authentication mechanism on smart-glass devices,
the technique will have to be evaluated over a large number of usage and
and over a large user base. Conducting such rigorous large-scale experiments
are typically infeasible in academic laboratory settings. We reserve such
large scale experiments for future work, and hope to accomplish through
industry collaborations.
