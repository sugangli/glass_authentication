\section{Discussion}\label{sec:disc}

In this study, we showed that head-movements have the potential to be used as
a reliable behavioral signature for user authentication.
We will now discuss some of the limitations that we identified from this work
and prospects for future work as below.

\iffalse
\subsection{Reliability}
Our work in this paper shows that head-movements are distinctive and
repeatable in controlled settings. However, in reality, the behavior of
head-movement signatures over chaotic settings will be a key factor to decide
on the effectiveness of this approach. Our work only evaluates the case when a
user is in a stationary setting when attempting to login, such as when sitting
on the chair or standing still. The performance of this approach in realistic
mobile settings such as walking or in a vehicle is yet unknown. It is also
unclear if the head-movement patterns are repeatable in such a mobile environment, or
if the ambiguities of vehicle motion versus head-motion can be separated. 
%%For example, we don't know whether a person's
%%head-movement signature will be the same no matter whether she is sitting,
%%standing still, walking, running, or driving.
%%The most important concern is the reliability of human head-movements. Even
%%though we have shown that head-movements are rather distinctive and
%%repeatable
%%in a controlled setting, it is yet unclear how it will perform in realistic,
%%but chaotic settings. For example, we don't know whether a person's
%%head-movement signature will be the same no matter whether she is sitting,
%%standing still, walking, running, or driving.
Similarly, a person's head-movement signature may also depend on the mood/energy level of
the person; for example, a fresh and energetic user may provide significant
head-movements as compared to a sick or tired user whose signatures may not
even be detectable. Inconsistencies in the accelerometer sensor such as drift and temporal bias can
significantly affect the nature of inferred head-movement signature.
Head-movements, on the other hand, may also evolve over time for a person
which call for periodic calibration of the system and/or the training data.
%To address these two temporal changes, we may need to periodically
%recalibrate
%the sensors and dynamically adjust our training data to reflect new movement
%trends.
While reliability metric was out of the main scope of this paper, we are keen
to address the same in future work.
\fi

\subsection{Power consumption}

\begin{table}
\begin{tabular}{lccc}
\hline
Component    & Power Consumption (mW) & Duration (s)       \\ \hline\hline
Sensor  & 29                    & 10       \\
Speaker & 410                    & 10      \\
CPU      & 1600                   & 14.4 \\ \hline
\end{tabular}
\caption{Power consumption on Google Glass of components relevant to 
\systemname.  The CPU (running at 
maximum frequency) power consumption includes that of the heads-up display 
screen being ON as well. Duration marks the time for which
component was ON during the a 10 sec music cue length trial}
\label{tab:pow}
\end{table}

Google Glass is an example of a wearable device that is heavily battery power 
constrained. Measuring the power consumption of the Glass's battery is a 
challenging task as that requires physically dismantling the device. 
We refer to the measurement paper on Google Glass by Robert et. 
al~\cite{likamwa2014draining} for the power consumption of the 
key components relevant to \systemname~implementation on Glass: the speaker 
for music cue playback, the accelerometer sensor and the CPU being ON during 
the entire authentication process. We report the relevant numbers in 
Table~\ref{tab:pow}. While the high CPU power consumption may not necessarily 
be surprising, the speakers also extrude considerable energy from the battery. 
We note that one possible solution for future consideration would be to play 
the music cue as intermittent notes over the duration, for example a ping or a 
beat sound periodically, where the speaker would be switched ON only during 
playback.

\subsection{Is this secure?}
An authentication system must have an effective protocol ensuring security of 
the authenticating user's data. Our system runs an implicit
authentication protocol where the user is given a finite set of (calibrated) 
music tracks to pick, based on which the user makes head-movements that are 
used as unique signatures for authentication.
Our design assumes implicit security of the user's data, as a
user voluntarily accepts the enforcement of conducting head-movements  
in response to the music. It is arguable that such enforcements are an 
integral part of most commonly used authentication systems; for example, 
typing a password, swiping the finger on the fingerprint sensor, approving of 
the camera recognizing the face. In all these cases the user is aware 
that he/she is inputing data into the system for authentication.

One way of compromising security would be a successful spoof of the 
head-movement by an adversary. For example, head-movements from an authorized 
user may be imitated by an adversary attempting to login to the device. 
%However, this requires that the wearable device is physically accessible to 
%the adversary. Since head-worn devices are at the visual field of view of the 
%user the chances of it getting lost or being stolen will be lower than other 
%wearable devices such as smart-watch or smart-necklaces. 
If the head-movements from the user is regular (such as a nod), it may be 
easily imitated as opposed to a random head-shake such as a head-bang.
To understand the effect of imitation on the accuracy of authenticating a user 
to \systemname, we conducted an experiment (under the same set up as  
described in section~\ref{sec:results}) where 29 volunteer participants 
were asked to imitate the head-nod movements of one user (one of the authors) 
who was trying to authenticate to the device using a 10 sec music cue. A total 
of 30 trials were conducted of which 10 samples were used as test data and 20 
for training. Our evaluations of this dataset resulted in reasonable accuracy 
values of, an 
EER of 7.2\% and a balanced 
accuracy $BAC = 1 - ((FAR+FRR)/2)$ of 94.5\% for the authorized user. Our 
results indicate that attacking the system through imitation of a simple 
head-gesture can still be challenging.

%A seamless design
%would ensure that the system captures even the slightest of the subconscious
%head-movements in the event that the user does not make any enforced
%head-movements. In such cases the head-movement signatures will have to be
%much more elaborate with multiple attributes that correspond to the different
%realistic use-cases of the system.

\subsection{Multi-Modality}
Inconsistencies in the accelerometer sensor such as drift and temporal bias can
significantly affect the nature of inferred head-movement signature.
Head-movements, on the other hand, may also evolve over time for a person
which call for periodic calibration of the system and/or the training data.
The array of motion sensors (accelerometer, gyroscope, inertial measurement 
unit) open up opportunities for multi-modal motion sensing. For example,
in Glass, accelerometer data can be combined with gyroscope measurements to 
provide multi-dimensional head-movement features that can improve the quality 
of the inferred signatures. Head movements can also be combined with other 
body movements to generate valuable, reliable signatures for authentication. 
%Additionally, head-movement is just one type of body movements, and we can
%investigate other types of body movements as well.
For example, through a simple test experiment using the Google Glass
infra-red (IR) light sensor (we had to root the Google Glass to access the
IR sensor unit) we observed that the blinking and winking patterns of users in
response to the music stimulus were reasonably differentiable among users.
Such patterns may also independently serve as another biometric that can be
used for authentication purpose, or can be combined with head-movements for better results.
Recent studies have shown that heart beat or pulse can also serve as reliable
biometric for authentication purposes~\cite{hernandezbioglass,nymi}.
We reserve such potential enhancements to our system for future implementation.
%A recent study has shown that Google glass can detect human heart
%beat~\cite{hernandezbioglass}; heartbeats can thus be
%used as another biometric.
%If extra hardware can be introduced, then more movement patterns,
%such as eye movement, can be leveraged as well.

\iffalse
\subsection{Seamless Protocol}
An authentication system must have an effective protocol for authenticating
users seamlessly to their device. Our system runs a simple authentication
protocol where the user is given a finite set of (calibrated) music
tracks to pick, and based on which the user makes head-movements in response.
Our design assumes that the user voluntarily accepts the enforcement of the
requirement of head-movements in response to the music. A seamless design
would ensure that the system captures even the slightest of the subconscious
head-movements in the event that the user does not make any enforced
head-movements. In such cases the head-movement signatures will have to be
much more elaborate with multiple attributes that correspond to the different
realistic use-cases of the system.
%Our protocol can consist of several steps. In the first step, the user will
%be
%asked to choose a user name from all the legitimate users of the device.
%Then,
%the user will be asked to select the favorite music track of the claimed
%user.
%If the selection is correct, the device will play the music and ask the user
%to move along (including head movement, eye blinking/winking, etc). ***YZ:
%what else??? ***
\fi

\iffalse
\subsection{Processing/Battery Power Constraints}
Battery power consumption and computing power are very important parameters
for consideration when optimizing a design to accommodate to wearable devices.
Wearables usually have serious resource limitations, especially in terms of
processing power and battery power.
This paper, addresses these concerns through optimization strategies in the
head-movement signature classification stage of the proposed authentication
algorithm. For wearables it is important that such optimization strategies are
taken to next levels until a roadblock is reached.
%In this paper, we have considered a set of
%optimization techniques to reduce the processing demand as well as power
%consumption, e.g., testing against top $K$ samples instead of  the entire
%training set, using thresholding-based classifier instead of SVM classifier.
For example, one such strategy extension in this work can be that, after a
short duration, before the entire music is played, if it is found that a
user's movement does not match the signature of the claimed user
with sufficient confidence level, then the on-site classification may be
terminated  instead of waiting for the entire duration to yield the rejection.
Another example, may include cyber-foraging strategies to offload heavy
computation tasks, such as classification, to the user's Bluetooth paired
smartphone.
\fi

\subsection{Large Scale User Study}
To be adopted as a primary authentication mechanism on smart-glass devices,
the technique will have to be evaluated over a large number of usage and
and over a large user base. Conducting such rigorous large-scale experiments
is typically infeasible in academic laboratory settings. We reserve such
large scale experiments for future work, and hope to accomplish through
industry collaborations. We will, however, be releasing our data-sets to the 
public in the near future.

