\section{Introduction}\label{sec:intro}
After generations of technological revolutions, from wired to wireless communications, stationary to mobile machines, and large-sized to hand-held devices, we are now witnessing what can be deemed as the
next phase of mobile technology: widespread {\em wearable computers}. Research in
wearable computers can be dated back at least to the 1980s when Steve Mann
experimented with backpack computers and developed a prototype of heads-up-display goggles~\cite{mann1997wearable}. Thanks to the advances in hardware miniaturization technology, cheap sensors/processor
chips, and low-power sensing/computing, today, wearables are now available
off-the-shelf and on the way to become an integral part of human
lives~\cite{googleglass,smartwatch,fitbit}. In this proposal, we focus in this proposal, we focus on special-purpose wearable devices that are worn close to the body as opposed to the more general-purpose computing/communication platform like smartphones or tablets. Examples are smart glasses, smart wristbands, and smart watches, smart jewelry, and devices embedded in clothing like jackets or shoes. Such devices are typically very small and impose severe resource constraints.

With the proliferation of such wearable devices, they can be expected to be subjected to malicious attacks and preserving the security and privacy of these devices will become increasingly important.
Much of the collected data on such devices is personal in nature and often relates to the user's health. Any security and privacy solution for such devices, however, also has to strike an appropriate balance with user convenience, especially as users are interacting with an increasing number of such specialized devices. A fundamental building block for safeguarding the security and privacy of user data acquired on or accessed through wearable devices are user authentication techniques, since many solutions are only effective as long as the device itself is authenticated to the right user/owner.

Authentication on most commercially available wearable devices today~\cite{fitbit, smartwatch} relies on an indirect mechanism, where users can log in to their wearables through their phones. This requires the wearable device to be registered and paired to the user's mobile device, which makes it inconvenient as the user has to carry both devices. The security of this approach is also in question as it increases the chance of hacking into both the devices if either of the devices are lost or stolen. Some devices including Google Glass~\cite{googleglass} and FitBit's health tracker~\cite{fitbit} also allow linking the device to online accounts instead of the phone for user's convenience; however, this does not add any security benefit. Indirect authentication remains a dominant paradigm for wearables despite these fundamental shortcomings because these devices are \emph{seriously resource-constrained} in many aspects: battery power, computational and storage capabilities, and input/output methods. As a result, typical authentication methods designed for more powerful devices can not be directly applied and must operate indirectly through a paired smartphone or other more capable device. In this proposal, however, we take the viewpoint that wearables will become more independent units that have to maintain security guarantees without such paired devices and we seek to develop suitable \emph{direct authentication} methods that are both accurate and light-weight.

Before we design direct authentication methods for wearable devices, let us first consider the available solutions for other mobile systems, especially smartphones and tablets. Broadly speaking, the two most commonly used authentication methods on mobile systems are arguable password-based methods (with their variants) and biometric-based methods. However, we argue that neither of these two methods is really suitable for wearable devices. Typing passwords or drawing swipe patterns on wearable devices can be quite cumbersome due to their small input/output units, if they do have a touch sensor at all. Collecting and recognizing physiological biometrics (such as DNA, fingerprint, hand/finger geometry, iris, odor, palm-print, retinal scan, voice, etc.) requires specialized sensing hardware and processing resources that add cost, and many of these sensors are larger than the size of wearables themselves.

This project therefore focuses on a third class of direct authentication methods: relying upon the uniqueness of human behavior characteristics such as human walking gait, arm swings, typing patterns, body pulse beats, eye-blinks, etc. This way of authenticating users is often referred to as \emph{behavioral} biometrics, and existing work has largely studied it in the context of authenticating smart phones and tablets~\cite{rahman2014bodybeat,cornelius2014wearable,stevenage1999visual,okumura2006study,monrose2000keystroke,jorgensen2011mouse,bo2013silentsense,de2012touch}. The main advantage of using behavioral biometrics for mobile devices is that the signatures can be readily generated from raw data of built-in sensors such as motion sensors, camera, microphones etc. Considering that cameras and microphones, as well as vision/audio processing algorithms, are quite energy-hungry, we thus focus on those behavioral biometrics that can be easily captured by sensors that require less power consumption, such as accelerometer and gyroscope. More specifically, we propose to authenticate wearable devices to users based on one type of behavioral characteristics: our unique body movement patterns and their dependence on external stimuli that wearable devices can generate, such as vibrations and music.

Body movement patterns have long been used by us humans to discriminate between people. By watching how a person walks, dances, waves her hands, we can often recognize the person from afar. This is because human body movements are \emph{distinctive} and \emph{repeatable}.  Achieving the same through wearables, however, is not straightforward and poses significant research challenges: it is unclear whether these seriously-constrained devices are able to capture the movement patterns, process the data, and quantify the uniqueness of each user's behaviors. Moreover, each device will have only a limited view of body movements, dependent on its mounting position on the human body. In this proposal, we set out to conduct a holistic study of wearable authentication through body movements and to design an accurate, robust and light-weight authentication system. A key distinguishing feature of our work is that we will also consider stimuli that wearable devices can provide to design challenge-response inspired mechanisms, particularly stimuli that are difficult to observe even for the closest adversaries. For example, we can use fast-tempo music through earbuds to stimulate movements and to make such free-style movements more repeatable. Our preliminary investigations show that under very controlled experiments, music-stimulated body movements have great potential to be used to authenticate users to their wearable devices. In order to develop a full-fledge authentication system that works in realistic settings, we propose several techniques to maximize the authentication accuracy and robustness by trying to increase the information encoded in sensor signals, exploiting multiple movements and/or multiple sensors, learning how to authenticate users in mobile settings, and designing movement pattern-based ``passwords'', as well as minimize its power consumption and processing requirement by carefully selecting features and classifiers, pipelining authentication operations and dynamically adjusting the data sampling rate. We will consider features of movements pattern that are more biometric in nature as well as those that depend more on user knowledge. For example, a list of motions associated with different prompts/stimuli can take the place of a 'password'.

Our research involves a coordination of algorithm design and system evaluation, ultimately involving the construction of a realistic authentication system that can efficiently run on wearable devices. Our project also involves an important curriculum development effort. We intend to train next-generation workforces in the rapidly-growing wearable and mobile computing field, as well as recruit youth and women into research in this field.


