\section{Results from Prior NSF Projects}\label{sec:prior}\vspace{-8pt}

\noindent \textbf{Y. Zhang} has, in the last 5 years, been the PI on NSF grants:
(i) CNS-0546072,  ``CAREER:PROSE: Providing Robustness in Systems of Embedded Sensors'', \$484K,  7/1/06 - 6/30/11;
(ii) CNS-0831186, ``CT - ISG: ROME: Robust Measurement in Sensor Networks'', \$400K, 09/01/08-08/31/11;
and co-PI on NSF grants
(iii) CNS-0910557, ``TC:Large: Collaborative Research: AUSTIN-- An Initiative to Assure Software Radios have Trusted Interactions'', \$410K, 9/1/09 - 8/31/12, (iv) CNS-1040735, ``FIA: Collaborative Research: MobilityFirst: A Robust and Trustworthy Mobility-Centric Architecture for the Future Internet'', \$2.73M, 9/1/10 - 8/31/13, and (v) CNS-1423020, ``NeTS: Small: Transmit Only: Cloud Enabled Green Communication for
Dense Wireless Systems'', \$4.98M, 9/1/14 - 8/31/17.
The \emph{intellectual merit} from these research projects have led to several new contributions in the past 5 years to the areas of sensing and Internet of things~\cite{zan-etal:mdm10,sun2012association,sun2011improved,sun2012boomerang,moore2013building}, unobtrusive human context learning ~\cite{xu2012improving,xu2012towards,xu2013crowd++,xu2013scpl}, network virtualization~\cite{bhanage2011virtual,bhanage2011experimental}, and next generation Internet architecture design~\cite{vu2012dmap,sun2011improving,zhang2012content,liu2013secure,zhang2013using,li2013mobile,li2012popularity}. \emph{Broader Impact}: Dr. Zhang has advised 7 Ph.D. students,  has created Owl Platform (www.owlplatform.com), a low-power smart home sensing system based on results from prior NSF grants, has developed the global name resolution service prototype for the GENI national testbed, has redesigned the Computer Architecture, Database Systems, and Performance Evaluation courses at Rutgers-ECE. She is currently working with four Ph.D. students.

\textbf{Marco Gruteser} is an Associate Professor at WINLAB Marco Gruteser and has served as PI and Co-PI on several NSF projects in the areas of networking, vehicular applications, and privacy. As PI for a location privacy
project~\cite{nsf-ct-anonymity} he has also developed the path cloaking methods for
protecting privacy of time-series location traces~\cite{1315266_hoh_privacy_gps,
hoh06:_enhan_privac_preser_anony_locat} as well as the virtual trip line privacy
mechanisms and a corresponding secret-splitting architecture for protecting
privacy in probe vehicle-based automotive traffic monitoring
systems~\cite{hoh_virtual_trip_lines,1717364_hoh_privacy_traffic_monitoring}.
All together, ten PhD students have to date completed their degrees and the projects lead to deep industry collaborations with General Motors, NEC Labs, as well as product
development impact at Nokia with privacy technologies, and outreach to the general public by featuring into online, radio, and television media (e.g., MIT Technology Review, NPR, and CNN TV).
