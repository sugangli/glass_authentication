\section{Curriculum Development Activities and Broader Impact}\label{sec:education}\vspace{-6pt}

The proposed research consists of an important curriculum development effort. The PIs are planning to create a graduate level seminar course, entitled ``Security and Privacy Issues for Wearable Computing,'' with a special focus on issues such as data security and privacy, user authentication, and ultra low-power security for wearable devices. Students will read recent papers from leading conferences or journals in related fields such as ACM CCS, IEEE Symposium on Security and Privacy, Usenix Security, ACM Mobisys, IEEE Transactions on Mobile Computing, etc. The reading list will include enabling wearable technologies, their security and privacy concerns, as well as emerging applications for wearable devices. For this course, students will collaboratively work on research-oriented open-ended projects that focus on designing and evaluating security/privacy preserving algorithms for emerging wearable devices and applications.


%\section{Broader Impact}\vspace{-6pt}
%The proposed project has a broad and profound impact on many aspects of society, and can contribute to the achievement of societally relevant outcomes.

\vspace{6pt}\noindent\textbf{Preparing the next-generation workforce for a rapidly growing industry.} Smart sensors and wearable devices are predicted to be the next computing platform that will seamlessly weave into our everyday life. In fact, a large number of startups are emerging in the general areas of wearable technology~\cite{nymi,scheirer1999expression,di2005magic,choudhury2002sociometer,meyerhoff1993line,starner2000gesture,farringdon1999wearable,led2004design,hung2004wearable,mistry2009sixthsense,giansanti2008assessment}. To respond to this trend, it is important to thoroughly understand its impacts on people's security and privacy, and to be equipped with technologies to address these concerns. The proposed research and education activities will play the critical role of preparing the next-generation workforce in this promising and rapidly growing area.

\vspace{3pt}\noindent\textbf{Engaging Undergraduates and the Youth in Research:} As part of the broader impacts, the PIs plan to involve undergraduate students and talented students from local high-schools. Prof. Zhang has served as the faculty advisor for the Eta Kappa
Nu honor society for over 5 years. She has already worked with four undergraduate students from the honors program on related research projects, with publications coming out of several of these efforts (e.g.~\cite{jsspp03,sensorfusion05,xu:wise04,xu:mobihoc05}).

The PIs will also actively engage high school students through summer research experiences in this project. In particular, the team will accept students from the Liberty Science Center's Partners in Science program and the WINLAB summer research program. Third, the Rutgers team plans to accept an academic year intern from Bergen County Academies, a NJ magnet school that operates an intern program for its high school students and to identify appropriate independent study topics that allow undergraduate students to participate during the academic year. PI Gruteser will draw from his established relationships with these programs and his past mentees include students that have won prizes at regional high school science competitions and started undergraduate computer science programs at UC Berkeley and Stanford. Additionally, Prof. Zhang has been engaged in an outreach to local high school students (Highland Park and Franklin High Schools in NJ).

Moving forward, the team will continue to develop undergraduate research projects, offer summer internships to undergraduate and high school students, and encourage them to work in related fields.


\vspace{3pt}\noindent\textbf{Encouraging Female Students in Engineering:} Ensuring that Science, Technology, Engineering and Mathematics (STEM) reaches as broad of a base as possible is an important activity that the investigators intend to focus on.
Professor Zhang has been actively involved in The Society of Women Engineers at Rutgers
University, where a series of workshops and luncheon meetings are
organized each semester, to give talks that emphasize the important
leadership roles open to women in EE/CS. She has supervised three female Ph.D. students, more than ten female master students and two female undergraduate students (both went to graduate school). One of her former Ph.D. students is now a tenured Associate Professor at University of South Carolina.  In addition, she has
organized a panel that discussed the challenges a woman engineer may
encounter in her career, which was very well received among woman
engineering students. Prof. Gruteser has also advised four additional female PhD students and a recent female high school student mentee has just been accepted to Princeton University’s undergraduate computer science program. Moving forward, the PIs will continue their effort to encourage female students to engage in STEM.
\vspace{-8pt}
