\begin{abstract}
In this paper, we present the design, implementation and evaluation of a novel user authentication system, dubbed {\em Headbanger}, for
head-worn wearable devices by monitoring the user's unique head-movement patterns in response to an external audio stimulus.
Solutions today primarily rely on indirect authentication mechanisms through the user's smartphone, which can be cumbersome and susceptible to adversary intrusions. Biometric solutions, on the other hand, are subject to the availability of the specific sensors in the wearable unit. The proposed head-movement based user authentication effectively addresses these concerns, providing an accurate, robust, light-weight and convenient solution.

Through extensive experimental evaluation with 95 human subjects, we show
that our mechanism can accurately authenticate users with an average true acceptance rate of
<<<<<<< HEAD
95.57\% while keeping the average false acceptance rate of 4.43\%. We also show that, even simple head-movement patterns are robust against imitation attacks. Finally, we demonstrate  our authentication algorithm is rather light-weight: the overall processing latency on Google Glass is as low as 1.9 seconds.
=======
95.57\% while keeping the average false acceptance rate of 4.43\%. We also show that, even simple head-movement patters are robust against imitation attacks. Finally, we demonstrate  our authentication algorithm is rather light-weight: the overall processing latency on Google Glass is as low as 1.9 seconds.
>>>>>>> 9c5d62116f36ae94ef14d05c4da8bfc2ae6f4f10



%Using a head-worn personal imaging device as a running example and

\end{abstract}

%The recent years have seen a significant growth in popularity of
%smart wearable devices. This growth can be attributed to the advances in
%hardware miniaturization technology as well as economically affordable
%and energy efficient sensing and computing. While size, energy and cost
%constraints remain key motives for improvements in wearable computers'
%design, the aspect of user authentication has received relatively less
%attention. Wearable devices often collect and store sensitive data about
%users, and thus there is an obvious need to authenticate the right user to the
%device.